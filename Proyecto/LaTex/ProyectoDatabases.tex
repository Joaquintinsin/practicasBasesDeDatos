\documentclass[15pt]{article}
\usepackage[spanish]{babel}
\usepackage[utf8]{inputenc}
\usepackage[T1]{fontenc}
\usepackage{graphicx}
\usepackage{listings}
\usepackage{setspace}
\usepackage{url}
\usepackage{xurl}
\usepackage{caption}
\usepackage[colorlinks=true, linkcolor=blue, urlcolor=blue] {hyperref}

\begin{document}

\begin{titlepage}
    \centering
    \begin{figure}
        \centering
        \includegraphics[width=0.35\linewidth]{escudo_unrc.jpg}
    \end{figure}
    \vfill
    
    \huge
    \textbf{Proyecto de Bases de Datos}
    \vspace{0.5cm}
    
    \LARGE
    Trabajo práctico integrador
    \vspace{1cm}

    \Large
    Facultad de Ciencias Exactas, Físico-Químicas y Naturales\\
    Departamento de Computación y Matemática\\
    \vspace{0.5cm}
    
    \vfill
    \Large
    \textbf{Nicolle Rosatti, Tomás Rodeghiero, Joaquín Tissera}\\
    \vspace{0.25cm}
    Año 2024
    \vspace{1.5cm}
\end{titlepage}

\newpage

\section{Inciso 1 y 2}
    \begin{itemize}
        \item[1] Diseñar el diagrama de Entidades y Relaciones.
            \begin{center}
              \textbf{Diagrama E/R}
              
              \textit{Adjuntado a color en las hojas presentadas}
              \newline
            \end{center}
        \item[2] Realizar el pasaje del modelo E-R a Relacional (con claves foráneas).
            \begin{center}
                \textbf{Tabulación del modelo E/R}
            \end{center}
    \end{itemize}
\newline
Pelicula(\underline{\textbf{identificador}}, genero, idioma\_original, url, duración, calificacion, 

fecha\_estreno\_Argentina, resumen, titulo\_distribucion, titulo\_original, 

titulo\_espaniol)
\newline
Cine(\underline{\textbf{nombre}}, direccion, telefono)
\newline
Sala(\underline{\textbf{identificador}}, cant\_butacas, nombre\_cine)

\hspace{1cm} nombre\_cine Foreing Key a Cine
\newline
Funcion(\underline{\textbf{codigo}}, fecha, hora\_comienzo, identificador\_sala)

\hspace{1cm} identificador\_sala Foreing Key a Sala
\newline
Pais(\underline{\textbf{nombre}})
\newline
Persona(\underline{\textbf{nombre}}, nacionalidad)

\hspace{1cm} Director(\underline{\textbf{nombre}})  nombre  Foreing Key a Persona
    
\hspace{1cm} Actor(\underline{\textbf{nombre}})   nombre  Foreing Key a Persona
    
\hspace{2cm} Protagonista(\underline{\textbf{nombre}})    nombre  Foreing Key a Actor
        
\hspace{2cm} Reparto(\underline{\textbf{nombre}}) nombre  Foreing Key a Actor
\newline
\newline
OrigenProduccion(\underline{\textbf{identificador\_pelicula}}, \underline{\textbf{nombre\_pais}}, anio\_produccion)

\hspace{1cm} identificador\_pelicula	Foreing Key a Pelicula

\hspace{1cm} nombre\_pais	Foreing Key a Pais
\newline
Dirige(\underline{\textbf{identificador\_pelicula}}, \underline{\textbf{nombre\_director}})

\hspace{1cm} identificador\_pelicula	Foreing Key a Pelicula
    
\hspace{1cm} nombre\_director	Foreing Key a Director
\newline
Proyecta(\underline{\textbf{identificador\_pelicula}}, \underline{\textbf{codigo\_funcion}})

\hspace{1cm} identificador\_pelicula	Foreing Key a Pelicula
    
\hspace{1cm} codigo\_funcion	Foreing Key a Funcion
\newline
EsReparto(\underline{\textbf{identificador\_pelicula}}, \underline{\textbf{nombre\_reparto}})

\hspace{1cm} identificador\_pelicula	Foreing Key a Pelicula
    
\hspace{1cm} nombre\_reparto	Foreing Key a Reparto
\newline
EsProtagonista(\underline{\textbf{identificador\_pelicula}}, \underline{\textbf{nombre\_protagonista}})

\hspace{1cm} identificador\_pelicula	Foreing Key a Pelicula
    
\hspace{1cm} nombre\_protagonista	Foreing Key a Protagonista
\newline
\newline

\end{document}
